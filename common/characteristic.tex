\note{Актуальность: есть куча всего, пристальное внимание ведущих компаний}
Составление портфеля из рыночных инструментов -- задача, которую решает любой крупный инвестиционный фонд. Из-за меняющейся рыночной и политической ситуации не вся историческая информация адекватно описывает настоящую динамику доходностей активов. 

\note{Почему требуется исследование?}
В недавнее время появились компании, составляющие портфели не из рыночных инструментов, а из торговых стратегий. Этот подход имеет свои преимущества и недостатки: 
\begin{itemize}
	\item[$+$] Стратегия может реагировать на экзогенные шоки
	\item[$+/-$] База стратегий очень большая
	\item[$-$] Присутствуют транзакционные издержки
	\item[$-$] Есть риск непредсказуемого поведения стратегии
\end{itemize}

В связи с этим, традиционные подходы портфельной оптимизации неприменимы, большое количество компонент портфеля (порядка сотен тысяч) несопоставимо с периодом доступных наблюдений, при этом, сами алгоритмы придумываются людьми не находящимися в штате на соревновательной основе. Ведущие компании\footnote{Например \textbf{Quantopian LLC} (Официальный сайт: \href{https://www.quantopian.com}{https://www.quantopian.com})},
заинтересованы в адаптации существующих методов к специфике задачи оптимизации портфеля торговых стратегий.

\note{Какие требование выдвигаются}
Выдвигаются особые требования к методу портфельной оптимизации:
\begin{itemize}
	\item Учет специфики динимики алгоритмов для оценки будущих рисков
	\item Способность работать с большой базой алгоритмов при составлении портфеля
	\item Учитывать специфику происхождения алгоритмов
\end{itemize}

\note{Цель, конкретика}
Цель работы -- разработка методологии портфельной оптимизации способной работать в условиях избыточного многообразия стратегий, дающую адекватную оценку будущих рисков и максимально учитывающую экспертные знания о технологическом процессе. 

\note{Задачи, поподробнее}
Таким образом необходимо решить следующие задачи в исследовании:
\begin{enumerate}
\item Разработать математическую модель динамики торговых стратегий
\item Выделить адекватную задаче спецификацию модели
\item Реализовать модель на языке \texttt{python}
\item Протестировать модель на реальных данных
\item Предложить способ составления портфеля на основе модели динамики
\item Сравнить разработанный метод с существующим, используемым на практике
\end{enumerate}

Работа над созданием метода составления портфеля торговых стратегий была проделана совместно с компанией \textbf{Quantopian LLC}. Они предоставили доступ к своей базе алгоритмов, инфраструктуре и вычислительным мощностям. Также была получена информационная поддержка по специфике задачи. Существующий в компании метод построения портфеля торговых стратегий базируется на портфельной теории Марковица \citep{markovitz1959}. По соглашению о неразглашении конфиденциальной информации программный код, созданный в процессе исследования, принадлежит компании, некоторые детали реализации также защищены.  
