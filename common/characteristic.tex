\note{Актуальность: есть куча всего, пристальное внимание ведущих компаний}
Составление портфеля из рыночных инструментов -- задача, которую решает любой крупный инвестиционный фонд. Из-за меняющейся рыночной и политической ситуации не вся историческая информация адекватно описывает настоящую динамику доходностей активов.

\note{Почему требуется исследование?}
В недавнее время появились компании, составляющие портфели не из рыночных инструментов, а из торговых стратегий. Торговая стратегия заключается в использовании исторических данных о доходностях активов для выбора оптимальной структуры портфеля активов на каждый следующий период времени. Результат работы алгоритма выражается в получаемой доходности, а портфель из торговых стратегий -- это набор алгоритмов, которые будут в заданных долях делить имеющиеся для инвестирования средства.
Этот подход имеет свои преимущества и недостатки: 
\begin{itemize}
	\item Стратегия может реагировать на экзогенные шоки, $(+)$
	\item База стратегий очень большая, $(+/-)$ 
	\item Присутствуют транзакционные издержки, $(-)$
	\item Есть риск непредсказуемого поведения стратегии, $(-)$
\end{itemize}

В связи с этим, традиционные подходы портфельной оптимизации неприменимы: большое количество компонент портфеля (порядка сотен тысяч) не сопоставимо с периодом доступных наблюдений, при этом, сами алгоритмы придумываются людьми не находящимися в штате на соревновательной основе, создавая проблемы оценки истинной доходности. Последние исследования показали существование не только динамики волатильности доходностей торговых активов, но и корреляций между ними. Ведущие компании\footnote{Например \textbf{Quantopian LLC} (Официальный сайт: \href{https://www.quantopian.com}{https://www.quantopian.com})} заинтересованы в адаптации существующих методов к специфике задачи оптимизации портфеля торговых стратегий для достижения желаемого соотношения риск--доходность. Эта задача очень новая, на практике к ней выдвигаются особые требования к методу портфельной оптимизации:
\begin{itemize}
	\item Учет специфики динамики доходностей алгоритмов для оценки будущих рисков
	\item Способность метода работать с большой базой алгоритмов при составлении портфеля
	\item Учитывать специфику происхождения алгоритмов
\end{itemize}

Однако в научной литературе специфика построения портфеля именно из торговых стратегий не освещена. В связи с интересом индустрии и недостатком теоретических исследований, эта тема является актуальной и востребованной.

\note{Цель, конкретика}
Цель работы -- создание методов составления портфеля биржевых торговых стратегий, учитывающих априорные знания о распределении их доходностей.

\note{Задачи, поподробнее}
Таким образом необходимо решить следующие задачи:
\begin{enumerate}
	\item Разработать общую схему моделирования динамики торговых стратегий, которая учитывает особенности, связанные с их доходностями: возможная динамика корреляций, стохастическая волатильность, различные законы распределения до и после момента создания алгоритма. Это включает в себя создание трех спецификаций модели с учетом, без учета динамики корреляций и упрощенная -- без корреляций
	\item Предложить метод оценивания параметров модели, и, ввиду отсутствия готовых решений, реализовать его в виде программного продукта
	\item Предложить способ составления портфеля на основе модели динамики
	\item Основываясь на реальных данных выбрать и оценить адекватную спецификацию модели, при этом, заранее не известно, какие из предпосылок выполняются
	\item Сравнить разработанный метод с существующим, используемым на практике
\end{enumerate}
\note{структура}

Сформулированные цели и задачи определили струтктуры работы. В первой главе на основе обзора литературы делается критический анализ методов формирования портфеля торговых стратегий.
Во второй главе проводится выбор адекватного инструментария моделирования динамики доходностей, в частности, обосновываются преимущества использования байесовского подхода. Предлагаются различные модификации модели динамики доходности торговых стратегий, соответствующие альтернативным предпосылки (наличие или отсутствие корреляций между доходностями и их постоянство или изменчивость во времени). Предлагается общая общая схема составления портфеля торговых стратегий на основе модели динамики доходностей.
Третья глава носит практический характер. На языке \texttt{Python} реализуется предложенная модель динамики доходностей и на основе предварительного анализа данных выбирается адекватная спецификация модели. Сравнение результатов использования модели  для построения портфеля торговых стратегий с традиционной процедурой делается на основе процедур бутстрап и монте--карло.

Успешность проведения исследования во многом определилась информационной поддержкой компании \textbf{Quantopian LLC}, которая предоставила доступ к своей базе алгоритмов торговли и вычислительным мощностям. В соответствии с соглашением, авторские права на использование программного кода, созданного автором, принадлежат компании \textbf{Quantopian LLC}.
