\chapter*{Заключение}
\addcontentsline{toc}{chapter}{Заключение}

В ходе проведенного исследования решены следующие задачи:
\begin{enumerate}
	\item Предложена общая схема моделирования динамики торговых стратегий, которая учитывает особенности, связанные с их доходностями. Результаты исследования включают в себя реализацию трех спецификаций модели с учетом, без учета динамики корреляций и упрощенная -- без корреляций. Для оценки модели динамики корреляций предложена модификация базового метода DECO с использованием гауссовского процесса

	\item Все три спецификации модели динамики доходностей торговых стратегий были реализованы на языке \texttt{Python} с использованием специализированных библиотек для байесовского моделирования
	
	\item Была реализована процедура составления портфеля торговых стратегий методом монте--карло

	\item Эмпирический анализ позволяет сделать вывод о непостоянстве волатильности, и постоянстве корреляций доходностей торговых стратегий

	\item Проведенные эксперименты показали практическую значимость предложенного подхода для использования его в качестве инструмента для составления портфеля торговых стратегий.
	Портфель, основанный на предложенной модели оказывается эффективнее модели Марковица по ряду критериев (имеет большую обобщающую способность, более высокое Шарпа на экзаменационной выборке). Оптимизация матожидания полезности инвестора позволяет получить более устойчивый во времени портфель.

\end{enumerate}

Проведенное исследование подтверждает целесообразность использования байесовских методов в формировании портфеля торговых стратегий. Такой подход позволяет учесть априорные знания о модели. Портфель, построенный на симуляциях из нее, получается более устойчивым во времени и, как следствие, более доходным на экзаменационном периоде.

