\chapter{Заключение}
Байесовский подход -- мощный практический инструмент моделирования случайных процессов. Он позволяет учесть априорные знания о модели, делая ее устойчивой к выбросам. Не исключением являются рыночные показатели доходностей. В условиях ограниченной длины ряда, динамической волатильности, и этапа предварительного отбора алгоритмов появляется возможность выразить предпосылки не только структурой модели, но и с помощью априорных распределений.

Практические результаты подтверждают необходимость использования баейсовских методов. Модель обладает большей обобщающей способностью и позволяет более аккуратно оценивать будущую динамику временного ряда, принимая во внимание неопределенность в параметрах модели. При этом, оптимизация матожидания полезности инвестора позволяет получить более устойчивый во времени портфель.
