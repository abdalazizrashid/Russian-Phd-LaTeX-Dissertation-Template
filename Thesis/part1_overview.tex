\chapter{Проблема построения портфеля торговых стратегий на финансовом рынке}
\note{Для начала надо рассказать про актуальность задачи диверсификации.}
Рациональное инвестирование денежных средств в рыночные активы подразумевает взвешенную оценку риска инвестирования и ожидаемой доходности. 
\section{Торговые стратегии и диверсификация рисков}
На рынке присутствует огромное количество активов с разными свойствами. Все активы в большей или меньшей степени подвержены рискам, например:
\todo{Перечисленное не совсем верно, уточнить}
\begin{enumerate}
	\item рыночный риск -- непредсказуемое поведение рынка, влияющее на все или часть секторов
	\item риск ликвидности -- сложность продать или приобрести актив
	\item индивидуальные риски компаний
\end{enumerate}
Инвестора будем рассматривать как рационального агента, который имеет некоторую функцию предпочтений относительно активов, так как целью является практическое применение модели для формирование портфеля. 

\note{Из нескольких активов можно комбинировать портфель с приемлемым соотношением риск-доходность}
Дискретный выбор одного единственного актива может не быть оптимальным выбором с точки зрения инвестора. Поскольку разные активы имеют разное соотношение риска и доходности, имеет смысл выбирать тот взвешенный набор активов, который бы максимизировал заданную функцию предпочтений, диверсифицируя риск \citep{markovitz1959}.

\note{Дальше стоит сказать про то, какие бывают портфели вообще, рассказать про доступное пространство риск - доходность, так как читатель может быть не знаком с этой теорией}
С математической точки зрения всевозможные выпуклые комбинации активов в портфеле образуют доступное множество соотношений риск-доходность. Выбор оптимальной -- задача портфельной оптимизации. При этом, учитываются особенности совместного распределения доходностей активов. Например, в отличие от матожидания, дисперсия доходности портфеля не может быть получена линейной комбинацией дисперсий компонент в общем случае. 

\note{Про алгоритмы. Необходимо четко обозначить то, что такое алгоритм и что это набор заранее определенных правил, кем то придуманный. Не останавливаясь на том, как они создаются, это будет в следующей части. Цели создания алгоритмов}
Существуют различные работы подтверждающие факт непостоянства рыночной ситуации \citep{billio2003, koutmos2012}. Меняются ожидаемые доходности активов и риски связанные с каждым из них. Таким образом, может быть полезным периодически переоценивать ситуацию на рынке и менять структуру портфеля. 

\note{Адаптивный портфель под изменяющуюся структуру рынка.}
Заранее заданные правила пересмотра называют торговой стратегией (алгоритмом). В течении периода работы алгоритм составляет динамически меняющийся портфель. Есть надежда, что торговая стратегия позволит диверсифицировать риски, связанные с изменением рыночной конъюнктуры и, в тоже время, риски, связанные с отдельными компаниями \citep{lorenz2008thesis}. 

\note{Надежды на возможность предсказать движение рыночных сил тут нет, лишь попытка снизить риск}
Автономность алгоритма и абстракция от специфики отрасли или отдельной компании дают надежду на то, что в какой-то степени получится этих рисков избежать. Стоит заметить, что возникают риски специфичные именно для алгоритмов:
\begin{itemize}
	\item риск непредсказуемого поведения
	\item риск смещенной оценки статистических показателей
\end{itemize}
Эти риски возникают по причине того, что каждый алгоритм создается человеком, наблюдающим результат процесса своей деятельности (создание алгоритма). Будущее поведение алгоритма не определено

\note{Возможность создания бесконечного многообразия торговых стратегий, как следствие можно составлять портфель из них, правильно учтя их свойства}
Диверсификация этих рисков требует качественного подхода к анализу доходностей, которые генерируются алгоритмом. Одним из простейших подходов диверсификации будет формирование портфеля из торговых стратегий, которые будут в заданных долях делить имеющиеся для инвестирования средства.

\section{Особенности анализа доходностей торговых стратегий}
\note{Структурные сдвиги}
Первая важная особенность торговых стратегий состоит в том, что они создаются человеком, автором. Автору доступна информация о том, как бы действовал алгоритм на исторических данных. Эта процедура называется <<бэктест>>, с ее помощью можно получить огромное количество статистик для дальнейшего анализа эффективности стратегии. Ориентируясь на эту информацию можно составить стратегию, которая хорошо работает на исторических данных. 

К сожалению, на последующем периоде поведение алгоритма непредсказуемо. Это является структурным сдвигом для торговой стратегии. Все что было до момента создания принимать во внимание, конечно, стоит, но с большой осторожностью, доверять следует статистикам, полученным на периоде после создания.
 
\note{Динамика волатильности (часто наблюдается и вообще у активов)}
\citep{dumas1998} рекомендует учитывать непостоянство волатильности при анализе финансовых рядов. Оснований утверждать, что для стратегий волатильность доходностей постоянна во времени -- нет. Корректная оценка рисков, связанных с непостоянной волатильностью, требует модели волатильности.

\note{Изменение характера взаимосвязи, уточнив., что гипотеза основана на анализе просто активов}
В ряде исследований \citep{vaga1990, oral2017} был выявлен факт непостоянства корреляций между доходностями компаний во времени. Так, например, в период кризисов корреляции усиливается. Рынок сильно влияет на эти связи непредсказуемым образом, похожие эффекты могут наблюдаться и для торговых стратегий. Моделирование динамики взаимосвязей поможет должным образом учесть эти риски.
\section{Формирование портфеля торговых стратегий}

\note{Критика портфельной теории Mарковица в литературе}
Портфельная теория Марковица \citep{markovitz1959} -- большой прорыв в решении задачи портфельной оптимизации.
Тем не менее, есть ряд недостатков портфельной теории Марковица \citep{lorenz2008thesis}:
\begin{itemize}
	\item инвестирование однопериодное
	\item модель не учитывает особенности распределения, только первый и второй центральные моменты
	\item инвестор не меняет состав портфеля
\end{itemize}
\note{Что вместо марковица?}
\cite{lorenz2008thesis, bucciol2006} предлагают использовать идеи \cite{neumann1944} и максимизировать ожидаемую функцию полезности инвестора. Это позволяет обобщить портфельную теорию, оптимальный портфель Марковица выступает частным случаем при определенных предпосылках.
\note{еще несколько теорий}

\note{и финальная фраза о том, что нам нужен последний метод}
Последний подход позволит учесть особенности торговых стратегий, он и будет использован для составления оптимального портфеля. Задачи, которые необходимо решить в рамках этого подхода:
\todo{расписать подробнее}
\begin{enumerate}
	\item составить модель динамики доходностей торговых стратегий и учесть:
	\begin{enumerate}
		\item сдвиги
		\item динамику волатильности
		\item динамику корреляций
	\end{enumerate}
	\item оценить модель
	\item получить симуляции из оцененной модели
	\item применить методы выпуклой оптимизации для формирования портфеля
\end{enumerate}